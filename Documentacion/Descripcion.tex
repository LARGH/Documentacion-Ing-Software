%---------------------------------------------------------
\section{Descripción del Módulo de Torneos}

El Módulo de Torneos a desarrollar permitirá que el coordinador del torneo de fútbol lleve un control sistematizado de los torneos de fútbol rápido y soccer además, los capitanes de cada equipo participante podrán hacer el registro completo y consultas de cualquiera de los dos torneos ya mencionados.\\

Respecto a las actividades del Coordinador tenemos las funcionalidades de: crear torneos, publicar convocatorias, abrir  la sección de registros estableciendo fechas y horarios para el registro apropiado de los equipos participantes tomando en cuenta que se cuenta con un cupo de máximo 16 equipos por torneo, publicar itinerarios y resultados del torneo , aprobar solicitudes en cambios de horario y realizar las modificaciones pertinentes, además el coordinador podrá administrar y publicar la actualización de estadísticas y eventos o  anuncios de importancia.\\

Tal como en el proceso actual de inscripción, el capitán de cada equipo participante tendrá un rol importante en el proceso de registro y representación del mismo, ya que el usuario capitán será quién haga el registro tanto del equipo, como de los jugadores participantes, considerando un mínimo y máximo de 11 y 22 jugadores respectivamente en la modalidad de soccer y en la modalidad de fútbol rápido 7 y 14 respectivamente; el capitán será el responsable de la aceptación de fechas y horarios asignados y de las solicitudes de cambios de horario.  
Finalmente, la actualización de las estadísticas y resultados del torneo, la publicación de los calendarios correspondientes al torneo de fútbol rápido y soccer e información acerca de los equipos integrantes de dichos torneos será visible a cualquier usuario que ingrese al sistema.

%---------------------------------------------------------
