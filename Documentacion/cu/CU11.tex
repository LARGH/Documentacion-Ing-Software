% Copie este bloque por cada caso de uso:
%-------------------------------------- COMIENZA descripción del caso de uso.

%\begin{UseCase}[archivo de imágen]{UCX}{Nombre del Caso de uso}{
	\begin{UseCase}{CU11}{Asignar grupo}{
		Resumen: Este caso de uso es exclusivo del coordinador, en este podra visualizar la lista de equipos que se han registrado durante la convocatoria, y una vez que el equipo haya pagado la cuota de inscripción el coordinador podra aceptar al equipo en el torneo y lo asignara al grupo que este quiera, todo sujeto a disponibilidad.}
		\UCitem{Versión}{2.0}
		\UCitem{Actor}{Coordinador}
		\UCitem{Propósito}{Asignar un grupo a los equipos que hayan pagado su inscripcion.}
		\UCitem{Entradas}{Ninguna}
		\UCitem{Origen}{\IUref{IU10}{ Pantalla de inicio coordinador.}}
		\UCitem{Salidas}{
			\begin{Citemize}
				\item {\bf MSG11}``¿Estás seguro que quieres asignar al +”nombre del equipo” al grupo +grupo? Si/Cancelar.''
				\item {\bf MSG11.1}`` Se asigno al +”nombre del equipo”  al +grupo correctamente.''
			\end{Citemize}
			}
		\UCitem{Destino}{\IUref{IU11.0}{Pantalla de solicitudes pendientes.}}
		\UCitem{Precondiciones}{Haberse autenticado como coordinador}
		\UCitem{Postcondiciones}{Asignación de grupos a los equipos participantes.}
		\UCitem{Errores}{ }
		\UCitem{Tipo}{Caso de uso primario.}
		\UCitem{Observaciones}{El coordinador puede aceptar en cada torneo a lo mas a 16 equipos.}
		\UCitem{Autor}{Versión 1.0 Aguilera Ramirez Ruben Omar. Versión 2.0 Morales García Miguel Angel.}
		\UCitem{Reviso}{Versión 1.0 Morales García Miguel Angel. Versión 2.0 Hernández Gómez Ricardo Neftali.}
		\UCitem{Estatus}{ Acualizado de 1.0 a 2.0,  Versión 2.0 revisada.}
	\end{UseCase}
	\newpage
	\begin{UCtrayectoria}{Principal}
	\UCpaso[\UCactor] Presiona el \IUbutton{Solicitudes pendientes} de la  \IUref{IU10}{ Pantalla de inicio coordinador.}.
	\UCpaso Muestra las solicitudes de los equipos para el Torneo de Fútbol Asociacón y Rápido.
	\UCpaso[\UCactor] Selecciona el grupo al cuál sera asignado el equipo solicitante.
	\UCpaso[\UCactor] Da clic en el \IUbutton{Aceptar} (representado con un palomita) que se encuentra frente a cada solicitud \Trayref{A}.
	\UCpaso Muestra el mensaje {\bf MSG11-}``¿Estás seguro que quieres asignar al +”nombre del equipo” al grupo +Grupo? Si/Cancelar.''
	\UCpaso[\UCactor] Da clic en \IUbutton{Si} \Trayref{B}.
	\UCpaso Muestra el mensaje {\bf MSG11.1-}`` Se asigno al +”nombre del equipo”  al +grupo correctamente.''
	\UCpaso Regresa a la pantalla \IUref{IU11.0}{Pantalla de solicitudes pendientes.}.
	
	\Trayref{A}.
\end{UCtrayectoria}

	\begin{UCtrayectoriaA}{A}{Ignorar solicitud}
		\UCpaso[\UCactor] Pasa por alto la solicitud del equipo.
		\UCpaso[\UCactor] Da clic en cualquier opción del menú lateral de \IUref{IU10}{ Pantalla de inicio coordinador.}
		\UCpaso Termina el caso de uso.
	\end{UCtrayectoriaA}

	\begin{UCtrayectoriaA}{B}{Cancelar Asignación}
		\UCpaso[\UCactor] Da clic en el boton \IUbutton{Cancelar}
		\UCpaso Cierra el {\bf MSG11}.
		\UCpaso Vuelve a \IUref{IU11}{ Pantalla de solicitudes pendientes.}. 
		\UCpaso Termina trayectoria alternativa.
	\end{UCtrayectoriaA}
	

%-------------------------------------- TERMINA descripción del caso de uso.