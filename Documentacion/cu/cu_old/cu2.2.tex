% \IUref{IUAdmPS}{Administrar Planta de Selección}
% \IUref{IUModPS}{Modificar Planta de Selección}
% \IUref{IUEliPS}{Eliminar Planta de Selección}

% Copie este bloque por cada caso de uso:
%-------------------------------------- COMIENZA descripción del caso de uso.

%\begin{UseCase}[archivo de imágen]{UCX}{Nombre del Caso de uso}{
	\begin{UseCase}{CU2.2}{Eliminar Equipo}{
		Resumen: El representante, podra eliminar uno de sus equipos registrados.}
		\UCitem{Versión}{1.0}
		\UCitem{Actor}{Representante del equipo.}
		\UCitem{Propósito}{Elminiar un equipo del torneo.}
		\UCitem{Entradas}{Equipo a eliminar.}
		\UCitem{Origen}{Pestaña Gestionar Equipos de la Pantalla del Representante.}
		\UCitem{Salidas}{Ninguna.}
		\UCitem{Destino}{Ninguno.}
		\UCitem{Precondiciones}{Tener un equipo registrado.}
		\UCitem{Postcondiciones}{Registro de equipo borrado.}
		\UCitem{Errores}{}
		\UCitem{Tipo}{Caso de uso secundario.}
		\UCitem{Observaciones}{}
		\UCitem{Autor}{Ricardo Neftali Hernández Gómez.}
		\UCitem{Reviso}{Omar Aguilera Ramirez}
	\end{UseCase}
	\newpage
	
	\begin{UCtrayectoria}{Principal}
	\UCpaso[\UCactor] Presiona el botón \IUbutton{ X } que se encuentra delante de cada equipo registrado, de la sección ``Mis Equipos'' de \IUref{IU2}{Gestionar Equipos}.
	\UCpaso Muestra un mensaje de confirmación.
	\UCpaso[\UCactor]Puede presionar el botón \IUbutton{Si} \Trayref{A}  o el botón \IUbutton{Cancelar} \Trayref{B}
	\UCpaso Termina caso de uso.
\end{UCtrayectoria}
	
	\begin{UCtrayectoriaA}{A}{Confirmar.}
		\UCpaso Borra el equipo seleccionado de los registros del sistema.
		\UCpaso Regresa a la sección origen.
		\UCpaso Termina caso de uso.
	\end{UCtrayectoriaA}
	
	\begin{UCtrayectoriaA}{B}{Declinar.}
		\UCpaso Regresa a la sección origen.
		\UCpaso Termina caso de uso.
	\end{UCtrayectoriaA}

%-------------------------------------- TERMINA descripción del caso de uso.
