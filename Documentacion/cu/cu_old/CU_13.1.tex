% Copie este bloque por cada caso de uso:
%-------------------------------------- COMIENZA descripción del caso de uso.

%\begin{UseCase}[archivo de imágen]{UCX}{Nombre del Caso de uso}{
	\begin{UseCase}{CU13.1}{Generar reporte de horarios de juego}{
		Resumen: El coordinador podrá generar un reporte de los horarios (rol de juegos, tabla de posiciones). }
		\UCitem{Versión}{1.0}
		\UCitem{Actor}{Coordinador}
		\UCitem{Propósito}{Generar reportes}
		\UCitem{Origen}{Pestaña generar reportes de la Pantalla del Coordinador.}
		\UCitem{Salidas}{Reporte generado}
		\UCitem{Destino}{Ninguno.}
		\UCitem{Precondiciones}{Haber iniciado sesión como coordinador}
		\UCitem{Errores}{}
		\UCitem{Tipo}{Caso de uso primario.}
		\UCitem{Observaciones}{}
		\UCitem{Autor}{Aguilera Ramirez Ruben Omar.}
		\UCitem{Reviso}{}
	\end{UseCase}
	\newpage
	\begin{UCtrayectoria}{Principal}
	\UCpaso[\UCactor] Presiona el \IUbutton{Generar reportes} de la  \IUref{IUPrincipal}{Pantalla inicio coordinador}.	\Trayref{A}.
	\UCpaso Muestra una lista con las opciones de reportes disponibles .
	\UCpaso[\UCactor] Da clic en el \IUbutton{Generar} de la opción seleccionada.
	\UCpaso Genera un PDF con el reporte correspondiente.
	\UCpaso Caso de uso en contruccion
	
	\Trayref{A}.
\end{UCtrayectoria}

	\begin{UCtrayectoriaA}{A}{El coordinador cierra sesión}
		\UCpaso[\UCactor] Da clic en  \IUbutton{Cerrar sesión} en la \IUref{IU1_1}{Pantalla de inicio de coordinador}.
		\UCpaso Termina el caso de uso.
	\end{UCtrayectoriaA}

	
%-------------------------------------- TERMINA descripción del caso de uso.