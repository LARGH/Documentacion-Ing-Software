\section{Introducción}

Como parte de la Unidad de Aprendizaje de Ingeniería de Software, se propone realizar un sistema que permita automatizar la mayor cantidad de las actividades del Club de fútbol de la Escuela Superior de Cómputo, las cuales son llevadas principalmente por el Coordinador éste, lo cual produce el riesgo de que toda la administración y funcionamiento dependa de la disposición del mismo.\\ \\

Así pues, se presenta la documentación correspondiente de un sistema que permita a sus usuarios una rápida y eficiente organización de los torneos de fútbol rápido y de asociación, desde la planeación de las fechas y horarios de partidos, hasta el momento de premiación, cuando culminen dichos torneos.\\ \\

En la Escula Superior de Cómputo del IPN cada semestre se llevan a cabo un torneo de futbol soccer y futbol rápido, con la participación de 16 equipos por torneo, formados por la comunidad estudiantil y el público en general interesado en participar. Actualmente todas las actividades de administración del torneo (convocatorias, inscripciones, calendario, resultados, estadísticas, posiciones) son realizadas principalmente por el coordinador del Club de Futbol de la ESCOM.\\ \\

El coordinador del club ha comunicado su interés por la realización de un sistema que le permita automatizar la mayor parte de las actividades de administración de ambos torneos, ya que esto le permitiría disminuir su carga de trabajo en las actividades administrativas.\\ \\

Dada la situación anterior, se propone realizar un sistema que permita automatizar el registro de equipos de fútbol, el registro de los jugadores de cada equipo, la elaboración del calendario que mostrará las fechas en las que se jugarán los partidos de fútbol y registrar los marcadores finales de los partidos, esto para el torneo de fútbol en sus dos modalidades: rápido y soccer. El sistema también permitirá la visualización de las estadísticas y resultados de ambos torneos, para cualquier persona interesada en esa información.\\ \\

En este documento se presenta el análisis llevado a cabo por el equipo \texttt{IS\char`_3CM3\char`_EQ1}

que cubrirá los requisitos identificados, detallados en la sección de requisitos funcionales y no funcionales, para que el sistema sea aceptado por el coordinador del torneo de fútbol.\\ \\

Alguna de las métricas fundamentales en este proyecto son la usabilidad, flexibilidad y escalabilidad, ya que el diseño debe estar orientado a usuarios poco familiarizados con las interfaces de usuario en dispositivos móviles o computadoras, y en un trabajo futuro se puedan agregar módulos para contabilizar las horas de entrenamiento al formar parte del club, así como generar las respectivas constancias de la Unidad de Aprendizaje Electiva.\\ \\
